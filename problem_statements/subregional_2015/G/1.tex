\documentclass[11pt,a4paper,oneside]{article}

\usepackage[T2A]{fontenc}
\usepackage[utf8]{inputenc}
\usepackage[english,russian]{babel}
\usepackage[russian]{olymp}

\usepackage{graphics}
\usepackage{wrapfig}
\usepackage{amsmath}
\usepackage{amssymb}
\usepackage{epigraph}
\usepackage[T1]{fontenc}
\usepackage{upquote}

\newcommand{\qo}{\textquotesingle}
\newcommand{\qq}{\textquotesingle~}

\contest{ACM ICPC Kyrgyzstan Subregional 2015}{Бишкек}{1 ноября 2015 года}    

\binoppenalty=10000
\relpenalty=10000
\exhyphenpenalty=10000

\renewcommand{\t}{\texttt}

\createsection{\Note}{Примечание}

\renewcommand{\defaultmemorylimit}{256 мегабайт}

\begin{document}

\begin{problem}{AB-строки}{1 секунда}{32 мегабайта}



Рассмотрим строки,  состоящие из символов \qo A\qq  и \qo B\qq латинского алфавита. 

\textit{Подстрока} строки S - это строка, являющаяся непустой связной частью $S$.  Например, \qo A\qo, $~$\qo B\qo, \qo AB\qo, \qo BA \qo, \qo ABA\qq -- все различные подстроки строки \qo ABA\qq.
Заметим, что  любая строка является также своей подстрокой.

Подстроки \qo AA\qo, \qo BB\qo, \qo ABAB\qq называются \textit{специальными}. 

\textit{Преобразованием} над строкой $S$ назовем любую из следующих 2-ух операций:
\begin{enumerate}
\item Удаление специальной подстроки  из любой позиции $S$ (включая начало и конец). 
\item Добавление специальной подстроки   в любую позицию $S$. 
\end{enumerate}

Например, \qo A\textbf{ABAB}B\qq $\to$ \qo AB\qq, \qo \textbf{AA}B\qq $\to$ \qo B\qo,  \textbf{\qo BB\qq} $\to$ \qo\qq -- преобразования первого типа (заметим, что в результате преобразования может получаться пустая строка). \qo AB\qq $\to$ \qo A\textbf{BB}B\qo, \qo BA\qq $\to$ \qo BA\textbf{ABAB}\qq , \qo\qq $\to$  \qo AA\qq -- преобразования второго типа.

Даны 2 строки, $S_1$ и $S_2$,  состоящие из символов \qo A\qq и \qo B\qq латинского алфавита. 
Требуется определить, можно ли получить строку $S_2$ из строки $S_1$, применив некоторое конечное число (возможно, 0) преобразований над строкой $S_1$. Можно применять преобразования как первого, так и второго типа, в произвольном порядке.


\InputFile

В первой строке входных данных записано число $1 \le N \le 20$ - количество тестов.
В каждой из следующих $N$ строках через пробел записаны 2 непустые строки: $S_1$ и $S_2$ состоящие из символов \qo A\qq и \qo B\qq латинского алфавита.
Cуммарная длина строк $S_1$ и $S_2$ в каждом тесте  не превосходит $10^6$.
\OutputFile

Для каждого теста на отдельной строке выведите слово \texttt{Yes}, если из строки $S_1$ т можно получить строку $S_2$, применив некоторое конечное число (возможно, 0) преобразований (в произвольном порядке), и \texttt{No} в противном случае.

\Examples

\begin{example}%
\exmp{
4
ABAB AABB
AABABBBBBA AAAA
AA AB
ABABA BABA
}{
Yes
Yes
No
No
}%
\end{example}

\vspace{0.3cm}
\medskip\noindent
\textbf{Пояснение к примерам.}

В первом примере из строки \qo ABAB\qq можно получить строку \qo AABB \qq, применив последовательно преобразования 1)\qo \textbf{ABAB} \qq $\to$ \qo \qq, 2)\qo \qq $\to$ \textbf{AA},
3) \qo AA \qq $\to$ \qo AA\textbf{BB} \qq. 

Во втором примере из строки \qo AABABBBBBA\qq можно получить строку \qo AAAA\qq, например, применив преобразования  следующим образом:
1) \qo A\textbf{ABAB}BBA\qq $\to$ 'ABBAABBA'
2) \qo A\textbf{BB}A \qq $\to$ \qo AA\qo 
3) \qo AA\qq $\to$ \qo AA\textbf{AA}\qq

В последних двух примерах из первой строки невозможно получить вторую. 





\vspace{1.0cm}
\hfill \textit{Автор задачи: Евгений Балай}
\medskip\noindent




\end{problem}


\end{document}
%%% Local Variables:
%%% mode: latex
%%% TeX-master: t
%%% End:
